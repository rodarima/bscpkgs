\documentclass{article}
\usepackage{graphicx}

\begin{document}

\title{Example of Nix + \LaTeX{}}
\author{Rodrigo Arias Mallo}

\maketitle

\section{Nbody}
The nbody program has been executed with varying block sizes while the execution 
time $t$ is measured, as shown in the figure \ref{fig:nbody.test}.
%
\begin{figure}[h]
    \centering
		\includegraphics[width=0.45\textwidth]{@fig.nbody.baseline@/scatter.png}
		\includegraphics[width=0.45\textwidth]{@fig.nbody.baseline@/box.png}
		\caption{Nbody times with varying block size}
		\label{fig:nbody.test}
\end{figure}
%
The normalized time $\hat t$ is computed with the median time $t_m$ using $ \hat 
t = t / t_{m} - 1 $. It can be observed that the normalized times exceed the 
maximum allowed interval in most cases, except with the largest block sizes.

Once the experiment \texttt{exp.nbody.test} changes, the hash of the experiment 
program will change, therefore the plot will be updated and, lastly, this 
report.

\end{document}
